En Colombia, las empresas de servicios públicos todavía enfrentan un problema que parece simple pero que afecta a miles de personas diariamente: ¿cómo gestionar eficientemente las citas para atención ciudadana cuando los sistemas actuales ya no dan abasto? Electrohuila, la empresa de energía del departamento del Huila, procesaba más de 15,000 solicitudes cada mes usando métodos que dependían mucho del trabajo manual, lo cual inevitablemente generaba errores y frustraba a los usuarios. Este artículo documenta cómo desarrollamos un sistema completo de agendamiento de citas que cambió radicalmente esa situación. La solución que construimos tiene tres componentes principales que trabajan juntos: una app móvil para que los ciudadanos puedan agendar desde sus teléfonos, un portal web para que los administradores gestionen todo el sistema, y un backend robusto que mantiene todo funcionando correctamente. Decidimos usar tecnologías que, aunque modernas, ya han demostrado ser confiables en ambientes empresariales: .NET 9 para el servidor, Next.js 15 con TypeScript para el sitio administrativo, y .NET MAUI 9 para la aplicación móvil que funciona tanto en Android como iOS. La base de datos Oracle era un requisito de Electrohuila porque ya la usaban en otros sistemas. Durante el desarrollo aplicamos principios de Clean Architecture, lo que nos permitió trabajar en diferentes partes del sistema simultáneamente sin pisarnos los pies. El sistema final hace mucho más que solo agendar citas: valida automáticamente la disponibilidad en tiempo real, maneja la administración de empleados y sucursales, controla permisos de acceso de forma granular, respeta los días festivos regionales (que en Colombia tienen reglas particulares), y mantiene a todos informados mediante notificaciones instantáneas. Los resultados fueron mejores de lo esperado en varios aspectos. Lo que aprendimos es que cuando se combinan arquitecturas bien pensadas con metodologías de desarrollo iterativas, es totalmente posible crear aplicaciones empresariales que realmente mejoren la vida de las personas mientras reducen costos operativos. Este trabajo aporta evidencia concreta de que modernizar servicios públicos tradicionales no solo es viable sino necesario, especialmente en países donde la transformación digital apenas está comenzando.
