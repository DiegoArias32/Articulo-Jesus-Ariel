\section{Resultados}
El desarrollo del Sistema de Agendamiento de Citas PQR para Electrohuila resultó en una aplicación empresarial completamente funcional que cumple con todos los requerimientos establecidos. Los resultados se presentan en términos de funcionalidades implementadas, rendimiento del sistema, y calidad del código:

\subsection{Funcionalidades implementadas}
El sistema implementado incluye un conjunto completo de funcionalidades que cubren todos los aspectos de la gestión de citas PQR para Electrohuila:

\textbf{Módulo de gestión de citas:} Permite la creación, consulta, modificación y cancelación de citas. Implementa validación de disponibilidad en tiempo real, verificando que no existan conflictos de horarios y que el empleado esté disponible en la fecha y hora solicitada. Gestiona estados de citas (Agendada, Confirmada, Completada, Cancelada) con transiciones controladas. Permite filtrado de citas por fecha, sucursal, empleado y estado. Genera identificadores únicos para cada cita que los ciudadanos pueden usar para consultas posteriores.

\textbf{Módulo de administración de empleados:} Permite crear, modificar y desactivar empleados del sistema. Asigna empleados a sucursales específicas con horarios de atención configurables. Gestiona la disponibilidad de empleados considerando días festivos, vacaciones y ausencias. Implementa un calendario de disponibilidad que muestra gráficamente los horarios ocupados y disponibles. Permite la reasignación de citas cuando un empleado no está disponible.

\textbf{Módulo de roles y permisos:} Implementa un sistema robusto de control de acceso basado en roles (Administrador, Supervisor, Operador, Ciudadano). Define permisos granulares para cada funcionalidad del sistema (crear citas, modificar empleados, gestionar sucursales, ver reportes). Permite asignación de roles a usuarios con validación de permisos en cada operación. Implementa políticas de autorización que verifican permisos antes de ejecutar acciones críticas.

\textbf{Módulo de gestión de sucursales:} Permite administrar las diferentes sucursales de Electrohuila con información de ubicación, horarios de atención y capacidad. Asigna empleados a sucursales específicas. Configura días y horarios de atención por sucursal, permitiendo flexibilidad para sucursales con horarios especiales. Gestiona la capacidad máxima de citas por día y por sucursal.

\textbf{Módulo de días festivos:} Administra un calendario de días festivos nacionales y locales que afectan la disponibilidad de citas. Permite agregar, modificar y eliminar días festivos con validación de fechas. Aplica automáticamente restricciones de disponibilidad en días festivos, bloqueando el agendamiento de nuevas citas. Notifica a administradores sobre próximos días festivos que requieren planificación.

\textbf{Sistema de notificaciones en tiempo real:} Utiliza SignalR para enviar notificaciones push a usuarios conectados. Notifica cambios de estado de citas (confirmación, cancelación, modificación) tanto a ciudadanos como a administradores. Envía alertas sobre citas próximas (recordatorios 24 horas antes). Permite broadcast de mensajes administrativos a todos los usuarios conectados. Implementa reconexión automática cuando se pierde la conexión, garantizando que no se pierdan notificaciones.

\textbf{Aplicación móvil ciudadana:} Interfaz intuitiva para que ciudadanos puedan buscar disponibilidad, seleccionar fecha, hora y sucursal preferida, agendar citas proporcionando información básica. Permite consultar citas existentes utilizando el identificador único o número de documento. Implementa funcionalidad de cancelación de citas con confirmación. Muestra notificaciones sobre cambios de estado y recordatorios. Funciona offline para consulta de citas previamente descargadas.

\textbf{Portal administrativo web:} Dashboard con métricas clave (citas del día, citas pendientes, empleados activos, ocupación por sucursal). Interfaces para gestión completa de empleados, sucursales, roles, permisos y días festivos. Visualización de calendario con todas las citas agendadas. Generación de reportes básicos sobre utilización del sistema, empleados más solicitados, y sucursales con mayor demanda. Configuración del sistema incluyendo horarios generales, duración de citas, y parámetros de negocio.

\subsection{Rendimiento y escalabilidad del sistema}
Se realizaron pruebas de rendimiento para evaluar la capacidad del sistema de manejar carga concurrente:

\textbf{Tiempos de respuesta del API:} Las pruebas con Apache JMeter mostraron tiempos de respuesta promedio de 120ms para operaciones de lectura (GET) y 180ms para operaciones de escritura (POST, PUT). Bajo carga de 100 usuarios concurrentes, los tiempos de respuesta se mantuvieron por debajo de 500ms en el percentil 95, cumpliendo con los requisitos de rendimiento establecidos.

\textbf{Throughput del sistema:} El sistema demostró capacidad de procesar aproximadamente 500 peticiones por segundo en condiciones normales. Durante picos de carga simulada (200 usuarios concurrentes), el sistema mantuvo estabilidad procesando 300-350 peticiones por segundo sin errores.

\textbf{Rendimiento de consultas a Oracle:} La optimización mediante índices redujo el tiempo de consultas complejas de varios segundos a menos de 200ms en promedio. Las consultas para verificar disponibilidad (que incluyen múltiples JOINs y validaciones) se ejecutan en menos de 150ms. El uso de stored procedures para lógica compleja redujo el tiempo de procesamiento en un 40\% comparado con múltiples consultas individuales.

\textbf{Escalabilidad horizontal:} La arquitectura basada en API stateless facilita escalamiento horizontal mediante múltiples instancias del backend detrás de un balanceador de carga. SignalR se configuró con soporte para backplane (Redis), permitiendo que las notificaciones funcionen correctamente en un ambiente de múltiples servidores.

\subsection{Calidad y mantenibilidad del código}
La aplicación de principios de ingeniería de software resultó en un código base de alta calidad:

\textbf{Cobertura de pruebas:} Se alcanzó una cobertura del 72\% en pruebas unitarias de la lógica de negocio crítica. Las validaciones de disponibilidad, gestión de permisos, y cálculo de horarios cuentan con suites completas de pruebas automatizadas. Las pruebas de integración validan los flujos end-to-end más importantes del sistema.

\textbf{Complejidad del código:} El análisis estático del código mediante herramientas como SonarQube mostró complejidad ciclomática promedio de 5 por método, dentro de rangos aceptables. No se identificaron code smells críticos ni duplicación significativa de código.

\textbf{Documentación:} El código incluye comentarios XML en todos los métodos públicos del API, generando documentación automática con Swagger. Se documentaron decisiones arquitectónicas importantes en el repositorio. Se crearon diagramas de arquitectura y de flujo de datos para facilitar la comprensión del sistema.

\textbf{Mantenibilidad:} La separación clara de responsabilidades mediante Clean Architecture facilita la localización y corrección de bugs. La independencia de frameworks permite actualizar tecnologías específicas sin afectar la lógica de negocio. El uso de inyección de dependencias facilita la creación de mocks para pruebas y el reemplazo de implementaciones.

\subsection{Cumplimiento de requerimientos}
El sistema cumple completamente con los requerimientos funcionales y no funcionales establecidos:

\begin{itemize}
    \item \textbf{Requerimientos funcionales:} Todas las historias de usuario definidas fueron implementadas y validadas con stakeholders de Electrohuila.

    \item \textbf{Seguridad:} Se implementó autenticación robusta, autorización basada en roles, almacenamiento seguro de contraseñas, protección contra vulnerabilidades comunes (SQL injection, XSS, CSRF).

    \item \textbf{Usabilidad:} Las pruebas con usuarios reales mostraron alta satisfacción con la interfaz móvil y el portal administrativo. El tiempo promedio para agendar una cita es de 2 minutos.

    \item \textbf{Disponibilidad:} El sistema está diseñado para alta disponibilidad mediante manejo de errores robusto, reconexión automática, y degradación elegante ante fallos de componentes.

    \item \textbf{Rendimiento:} Los tiempos de respuesta cumplen con las expectativas de usuarios, proporcionando una experiencia fluida tanto en web como en móvil.
\end{itemize}

En síntesis, el proyecto resultó en un sistema empresarial completo, funcional, escalable y mantenible que demuestra la aplicación exitosa de principios modernos de ingeniería de software en un contexto de formación práctica del SENA.
