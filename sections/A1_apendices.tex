\appendix
\section{Herramientas de IA utilizadas}
Durante el desarrollo del Sistema de Agendamiento de Citas PQR para Electrohuila se utilizaron las siguientes herramientas de Inteligencia Artificial:

\begin{itemize}[leftmargin=*]
  \item \textbf{GitHub Copilot}: Asistente de código integrado en el IDE para generación de código en tiempo real, autocompletado inteligente de funciones, y sugerencias contextuales basadas en el código existente.
  
  \item \textbf{Claude Code}: Asistente técnico especializado para debugging, explicación de conceptos arquitectónicos complejos, y generación de código estructurado para .NET Core, Next.js y MAUI.
  
  \item \textbf{ChatGPT}: Consultas técnicas para validación de arquitecturas, resolución de problemas específicos, explicación de patrones de diseño, y generación de queries Oracle optimizadas.
  
  \item \textbf{Consultas técnicas especializadas}: Uso de modelos de lenguaje para explicación de conceptos específicos como Clean Architecture, CQRS, MVVM, Server Components, SignalR, Entity Framework Core con Oracle, y optimización de rendimiento.
  
  \item \textbf{Diseño y arquitectura}: IA para inspiración de diseños de interfaz modernos, búsqueda de patrones de UI/UX, validación de flujos de usuario, y recomendaciones de componentes reutilizables.
  
  \item \textbf{Aprendizaje continuo}: Uso de IA como mentor técnico disponible 24/7 para el equipo, proporcionando explicaciones pedagógicas de conceptos complejos y facilitando el aprendizaje de tecnologías empresariales.
\end{itemize}

\section{Stack tecnológico completo}
El proyecto fue desarrollado utilizando el siguiente stack tecnológico:

\subsection{Backend}
\begin{itemize}[leftmargin=*]
  \item .NET Core 8 (C\#)
  \item Entity Framework Core con Oracle Provider
  \item Clean Architecture + CQRS
  \item SignalR para tiempo real
  \item JWT Authentication
  \item AutoMapper para DTOs
  \item FluentValidation para validaciones
\end{itemize}

\subsection{Frontend Web}
\begin{itemize}[leftmargin=*]
  \item Next.js 14 con App Router
  \item TypeScript
  \item Zustand (estado global)
  \item Tailwind CSS + shadcn/ui
  \item React Hook Form
  \item SignalR Client
  \item Axios para HTTP
\end{itemize}

\subsection{Frontend Mobile}
\begin{itemize}[leftmargin=*]
  \item .NET MAUI (Android/iOS)
  \item MVVM Pattern
  \item CommunityToolkit.MVVM
  \item HttpClient
  \item SecureStorage
  \item Local Notifications
\end{itemize}

\subsection{Base de datos}
\begin{itemize}[leftmargin=*]
  \item Oracle Database
  \item Stored Procedures
  \item Modelo Relacional Normalizado (3NF)
  \item Índices optimizados
\end{itemize}

\section{Repositorio del proyecto}
El código fuente y la documentación completa del proyecto están disponibles para consulta académica. El proyecto incluye:

\begin{itemize}[leftmargin=*]
  \item Modelo entidad-relación de la base de datos Oracle
  \item Mockups en Figma del portal web y la aplicación móvil
  \item Código completo del backend API en .NET Core 8
  \item Código completo del frontend web en Next.js 14
  \item Código completo de la aplicación móvil en .NET MAUI
  \item Scripts SQL para creación de base de datos Oracle
  \item Documentación de arquitectura y patrones utilizados
  \item Guías de instalación y configuración
\end{itemize}

El repositorio puede consultarse en: \url{https://github.com/Electrohuila-PQR}
